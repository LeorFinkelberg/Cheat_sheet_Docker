\documentclass[%
	11pt,
	a4paper,
	utf8,
	%twocolumn
		]{article}	

\usepackage{style_packages/podvoyskiy_article_extended}


\begin{document}
\title{Практика использования и \\наиболее полезные конструкции \texttt{Docker}}

\author{\itshape Подвойский А.О.}

\date{}
\maketitle

\thispagestyle{fancy}

\tableofcontents

\section{Общие сведения о системе Docker}

\subsection{Установка}

Установить \texttt{Docker} можно с помощью менеджера пакетов \texttt{conda}

\begin{lstlisting}[
style = ironpython,
numbers = none
]
conda install -c conda-forge docker-py
\end{lstlisting}

На текущий момент без серьезных проблем \texttt{Docker} работает только на 64-битовом \texttt{Linux}.

Для нормальной работы на \texttt{MacOS X} или \texttt{Windows} потребуется дополнительно установить какую-либо виртуальную машину в полной комплектации или пакет \texttt{Docker Toolbox}:

\begin{itemize}
	\item для \texttt{MacOS X}: \url{https://docs.docker.com/toolbox/toolbox_install_mac/},
	
	\item для \texttt{Windows}\footnote{Поддерживается даже \texttt{Windows 7}}: \url{https://docs.docker.com/toolbox/toolbox_install_windows/}; после установки \texttt{Docker Toolbox} останется только запустить \texttt{Docker Quick Start Terminal}.
\end{itemize}

\subsection{Контейнеры}

\emph{Контейнеры} представляют собой средства инкапсуляции приложения вместе со всеми его зависимостями.

Поскольку \texttt{Docker} сам по себе не обеспечивает реализацию любого типа виртуализации, контейнеры всегда должны соответствовать ядру хоста -- контейнер на \texttt{Windows Server} может работать только на хосте под управлением операционной системы \texttt{Windows Server}, а 64-битный \texttt{Linux}-контейнер работает только на хосте с установленной 64-битной версией операционной системы \texttt{Linux} \cite{mouat:docker-2017}.

\subsection{Создание образов из Dockerfile}

\texttt{Dockerfile} -- это обычный текстовый файл, содержащий набор операций, которые могут быть использованы для создания Docker-образа.

Пример. Для начала создадим новый каталог и собственно Dockerfile

\begin{lstlisting}[
numbers = none
]
$ mkdir cowsay
$ cd cowsay
$ touch Dockerfile
\end{lstlisting}

Затем в созданный \texttt{Dockerfile} добавим следующее 
\begin{lstlisting}[
title = {\sffamily Dockerfile},
numbers = none
]
FROM debian:wheezy
MAINTAINER John Smith <john@smith.com>
RUN apt-get update && apt-get install -y cowsay fortune
COPY entrypoint.sh /
ENTRYPOINT ["/entrypoint.sh"]
\end{lstlisting}

Инструкция \texttt{FROM} определяет базовый образ ОС (это в данном случае \texttt{debian} с уточненной версией <<wheezy>>). Инструкция \texttt{FROM} является строго обязательной для всех файлов Dockerfile как самая первая незакомментированная инструкция.

Инструкция \texttt{MAINTAINER} просто определяет информацию, позволяющую связаться с автором образа.

Инструкция \texttt{COPY} копирует файл из файловой системы хоста в файловую систему образа, где первый аргумент определяет файл хост, а второй -- целевой путь.

Инструкция \texttt{RUN} определяет команды, выполняемые в командной оболочке внутри данного образа.

Комментарии к скрипту \texttt{entrypoint.sh}. Файл \texttt{entrypoint.sh} должен лежать в той же директории, что и файл \texttt{Dockerfile} и иметь содержание на подобие следующего
\begin{lstlisting}[
title = {\sffamily entrypoint.sh},
numbers = none
]
if [ $# -eq 0 ]; then
    /usr/games/fortune | /usr/games/cowsay
else
    /usr/games/cowsay "$@"
fi
\end{lstlisting}

Здесь конструкция \texttt{[...]} -- это форма\footnote{Есть еще вариант \texttt{test выражение}, но форма \texttt{[ выражение ]} более популярна} команды \texttt{test} для проверки различных условий. Последовательность символов \texttt{\$\#} -- встроенная переменная, обозначающая количество аргументов в командной строке. Последовательность символов \texttt{\$@} -- это все аргументы командной строки, а \texttt{"\$@"} -- все аргументы командной строки, заключенные по отдельности в кавычки \cite[\strbook{44}]{robbins:bash-2017}.

После сохранения необходимо сделать этот файл исполняемым при помощи команды\\ \texttt{chmod +x entrypoint.sh}.

Теперь можно создать образ на основе файла \texttt{Dockerfile}
\begin{lstlisting}[
numbers = none
]
docker build -t test/cowsay-dockerfile .
\end{lstlisting}

Здесь \texttt{test} -- имя репозитория, а \texttt{cowsay-dockerfile} -- имя образа.

После этого можно запускать контейнер, который строится на основе образа \verb|test/cowsay-dockerfile|
\begin{lstlisting}[
numbers = none
]
docker run test/cowsay-dockerfile /usr/games/cowsay 'Moo'
\end{lstlisting}



\section{Общие сведения о компьютерных сетях}

\subsection{Термины и определения}

\paragraph{{localhost}} (так называемый, <<локальный хост>>, по смыслу <<этот компьютер>>) -- стандартное, официально зарезервированное доменное имя для \emph{частых} (или что то же самое \emph{локальных}) IP-адресов\footnote{Уникальный сетевой адрес узла в компьютерной сети, построенной на базе стека протоколов TCP/IP} \emph{петлевого интерфейса}\footnote{Обычно используется термин \emph{loopback}, который описывает методы или процедуры маршрутизации электронных сигналов, цифровых потоков данных, или других движущихся сущностей от их источника и обратно к тому же источнику без специальной обработки или модификации} (диапазон \texttt{127.0.0.1} -- \texttt{127.255.255.255}). Использование IP-адреса \texttt{127.0.0.1} позволяет устанавливать соединение и передавать информацию для программ-серверов, работающих на том же компьютере, что и программа-клиент. Примером может быть запущенный на компьютере веб-сервер приложений, обращение к которому выполняется с этого же компьютера для веб-разработки на данном компьютере без необходимости выкладывать веб-программу в сеть Интернет, пока ее разработка не закончена. Традиционно IP-адресу \texttt{127.0.0.1} однозначно сопоставляется имя хоста \texttt{localhost}.

\paragraph{{порт}} целое неотрицательное число, записываемое в заголовках \emph{протоколов транспортного уровня} модели OSI (TCP, UDP, SCTP, DCCP). Используется для определения процесса-получателя пакета в пределах одного хоста (локального компьютера).

\section{Базовые концепции}

\subsection{Виртуальный сетевой интерфейс}

Все TCP/IP-реализации поддерживают loopback-механизмы, которые реализуют виртуальный сетевой интерфейс исключительно программно и не связанны с каким-либо оборудованием, но при этом полностью интегрированны во внутреннюю сетевую инфраструктуру компьютерной системы. Пожалуй самый распространенным IP-адресом в механизмах loopback является \texttt{127.0.0.1}. В IPv4 в него также отображается любой адрес из диапазона \texttt{127.0.0.0} -- \texttt{127.255.255.255}. IPv6 определяет единственные адрес для этой функции -- \texttt{0:0:0:0:0:0:0:1/128} (так же записывается как \texttt{::1/128}). Стандартное, официально зарезервированное доменное имя для этих адресов -- \texttt{localhost}.

Интерфейс loopback имеет несколько путей применения. Он может быть использован сетевым клиентским программным обеспечением, чтобы общаться с серверным приложением, расположенным на том компьютере. То есть если на компьютере, на котором запущен веб-сервер, указать в веб-браузере URL \texttt{http://127.0.0.1/} или \texttt{http:/localhost/}, то он попадает на веб-сайт этого компьютера.

\section{Наиболее полезные конструкции}

\subsection{Манипуляции с контейнерами}

Запустить контейнер с именем \texttt{leorcont}, создав сеанс интерактивной работы (\texttt{-i}) на подключаемом терминальном устройстве (\texttt{-t}) tty, и вызывать командную оболочку \texttt{bash} из-под ОС~\texttt{Ubuntu Linux}

\begin{lstlisting}[
numbers = none
]
docker run -it --name leorcont ubuntu bash
\end{lstlisting}

Запустить контейнер, а после остановки удалить сам контейнер и созданную на время его существования файловую систему

\begin{lstlisting}[
numbers = none
]
docker run --rm -it ubuntu bash
\end{lstlisting}

Перезапустить остановленный контейнер

\begin{lstlisting}[
numbers = none
]
docker start quizzical_wright
\end{lstlisting}

\subsection{Информация о контейнере}

Получить информацию о контейнере

\begin{lstlisting}[
numbers = none
]
docker inspect quizzical_wright
\end{lstlisting}

Вывести информацию о контейнере с использованием утилиты \texttt{grep}

\begin{lstlisting}[
numbers = none
]
docker inspect quizzical_wright | grep SandboxID
\end{lstlisting}

Вывести информацию о контейнере с использованием шаблона языка \texttt{Go} \url{https://metanit.com/go/web/2.2.php}

\begin{lstlisting}[
numbers = none
]
docker inspect --format {{.NetworkSettings.SandboxID}} quizzical_wright
\end{lstlisting}

Вывести список файлов в работающем контейнере. Для контейнеров \texttt{Docker} использует файловую систему \texttt{UnionFS}, которая позволяет монтировать несколько файловых систем в общую иерархию, которая выглядит как \emph{единая файловая система}. Файловая система конкретного \emph{образа} смонтирована как уровень \emph{только для чтения}, а любые изменения в работающем контейнере происходят на уровне с разрешенной записью, монтируемого поверх основной {файловой системы образа}. Поэтому \texttt{Docker} при поиске изменений в работающей системе должен рассматривать только самый верхний уровень, на котором возможна запись \cite{mouat:docker-2017}

\begin{lstlisting}[
numbers = none
]
docker diff quizzical_wright
\end{lstlisting}

Вывести список работающих контейнеров

\begin{lstlisting}[
numbers = none
]
docker ps
\end{lstlisting}

Вывести список всех контейнеров, включая остановленные (stopped)\footnote{Формально их называют контейнерами, из которых был совершен выход (exited containers)}. Такие контейнеры могут быть перезапущены с помощью \texttt{docker start}

\begin{lstlisting}[
numbers = none
]
docker ps -a
\end{lstlisting}

\subsection{Удаление контейнеров и образов}

Удалить контейнер 

\begin{lstlisting}[
numbers = none
]
docker rm quizzical_wright
\end{lstlisting}

Удалить несколько остановленных контейнеров можно следующим способом. Значение флагов: \texttt{-a} (все контейнеры), \texttt{-q} (вывести только числовой идентификатор контейнера), \texttt{-f} (фильтр), \texttt{-v} (все тома, на которые не ссылаются какие-либо другие контейнеры)

\begin{lstlisting}[
numbers = none
]
docker rm -v $(docker ps -aq -f status=exited)
\end{lstlisting}


% Источники в "Газовой промышленности" нумеруются по мере упоминания 
\begin{thebibliography}{99}\addcontentsline{toc}{section}{Список литературы}
	\bibitem{mouat:docker-2017}{ \emph{Моуэт Э.} Использование Docker. -- М.: ДМК Пресс, 2017. -- 354 с. }
	
	\bibitem{robbins:bash-2017}{ \emph{Роббинс А.} Bash. Карманный справочник системного администратора, 2-е изд.: Пер. с англ. -- СПб.: ООО <<Альфа-книга>>, 2017. -- 152 с.}
\end{thebibliography}

%\listoffigures\addcontentsline{toc}{section}{Список иллюстраций}


\end{document}
